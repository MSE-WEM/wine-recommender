Dans le cadre du cours de Web Mining, il est demandé de réaliser un projet de récupération et d'analyse de données.
Le sujet choisi pour ce dernier est la création d'un système de recommandation de vin en fonction d'une recette.


\section{Contexte}
Lorsque nous choisissons de préparer une recette, nous nous demandons parfois quel vin accompagnerait le mieux ce plat. 
En règle générale, les descriptifs de vins viennent accompagnés d'une recommandation du style "boeuf", "poulet", "poisson", etc.
Mais ces recommandations sont souvent trop générales et ne permettent pas de choisir un vin adapté à la recette. 
De plus, il est difficile de trouver des informations sur les vins, notamment sur les vins de qualité moyenne ou inférieure. 
C'est pourquoi nous avons décidé de créer un système de recommandation de vin en fonction d'une recette sous forme d'une application web.

Pour ce faire, deux sources de données seront utilisées:
\begin{itemize}
    \item \href{https://www.vivino.com/}{\textbf{Vivino}}: pour la récupération des données sur les vins.
    \item \href{https://www.marmiton.org/recettes/index/categorie/plat-principal/}{\textbf{Marmiton}}: pour la récupération des données sur les recettes.
\end{itemize}

Afin de traiter une quantité de données raisonnable, nous nous limitons aux plats principaux pour les recettes et aux vins rouges et blancs pour les vins.

\section{Objectifs}
La réalisation du projet est composé de trois parties principales:
\begin{itemize}
    \item \textbf{Récupération des données}: récupération des données sur les vins et les recettes.
    \item \textbf{Création de l'API}: création d'une API REST pour servir les données.
    \item \textbf{Création de l'application web}: création d'une application web pour la recommandation de vin.
\end{itemize}